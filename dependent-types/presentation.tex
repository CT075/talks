%%%%%%%%%%%%%%%%%%%%%%%%%%%%%%%%%%%%%%%%%
% Beamer Presentation
% LaTeX Template
% Version 1.0 (10/11/12)
%
% This template has been downloaded from:
% http://www.LaTeXTemplates.com
%
% License:
% CC BY-NC-SA 3.0 (http://creativecommons.org/licenses/by-nc-sa/3.0/)
%
%%%%%%%%%%%%%%%%%%%%%%%%%%%%%%%%%%%%%%%%%

%----------------------------------------------------------------------------------------
%	PACKAGES AND THEMES
%----------------------------------------------------------------------------------------

\documentclass[usenames,dvipsnames]{beamer}

\usepackage{lstautogobble}
\usepackage{ulem}
\usepackage{bussproofs}
\definecolor{eclipseBlue}{RGB}{42,0,255}
\definecolor{eclipseGreen}{RGB}{63,127,95}
\lstdefinestyle{150base}
{%
  basicstyle=\fontsize{9}{11}\ttfamily,
  columns=fixed,
  keepspaces=true,
  autogobble=true,
  breaklines=true,
  frame=single,
  numbers=left,
  numberstyle=\tiny,
  showstringspaces=false,
  stringstyle=\color{black},
  identifierstyle=\color{eclipseBlue},
  keywordstyle=\color{red},
  commentstyle=\color{eclipseGreen}
}
\lstdefinestyle{sml-specific}
{%
  morekeywords=
  {%
    bool, char, exn, int, real, string, unit,
    list, option,
    andalso, orelse, true, false, not,
    if, then, else, case, of,
    EQUAL, GREATER, LESS, NONE, SOME,
    let, in, end, local, val, rec,
    datatype, type, exception, handle,
    fun, fn, op, raise, ref, o, print,
    structure, struct, signature, sig, functor,
    include, open, use, infix, infixr,
    nil, as
  },
  morestring=[b]",
}
\lstdefinestyle{sml}{language=ML, style=150base, style=sml-specific}
\lstnewenvironment{code}[1][]
{\lstset{style=sml,stringstyle=\color{orange},numbers=none,#1}}{}

\mode<presentation> {

% The Beamer class comes with a number of default slide themes
% which change the colors and layouts of slides. Below this is a list
% of all the themes, uncomment each in turn to see what they look like.

%\usetheme{default}
%\usetheme{AnnArbor}
%\usetheme{Antibes}
%\usetheme{Bergen}
%\usetheme{Berkeley}
%\usetheme{Berlin}
%\usetheme{Boadilla}
%\usetheme{CambridgeUS}
%\usetheme{Copenhagen}
%\usetheme{Darmstadt}
%\usetheme{Dresden}
%\usetheme{Frankfurt}
%\usetheme{Goettingen}
%\usetheme{Hannover}
%\usetheme{Ilmenau}
%\usetheme{JuanLesPins}
%\usetheme{Luebeck}
%\usetheme{Madrid}
%\usetheme{Malmoe}
%\usetheme{Marburg}
%\usetheme{Montpellier}
%\usetheme{PaloAlto}
\usetheme{Pittsburgh}
%\usetheme{Rochester}
%\usetheme{Singapore}
%\usetheme{Szeged}
%\usetheme{Warsaw}

% As well as themes, the Beamer class has a number of color themes
% for any slide theme. Uncomment each of these in turn to see how it
% changes the colors of your current slide theme.

%\usecolortheme{albatross}
%\usecolortheme{beaver}
%\usecolortheme{beetle}
%\usecolortheme{crane}
%\usecolortheme{dolphin}
%\usecolortheme{dove}
%\usecolortheme{fly}
%\usecolortheme{lily}
%\usecolortheme{orchid}
%\usecolortheme{rose}
%\usecolortheme{seagull}
\usecolortheme{seahorse}
%\usecolortheme{whale}
%\usecolortheme{wolverine}

%\setbeamertemplate{footline} % To remove the footer line in all slides uncomment this line
%\setbeamertemplate{footline}[page number] % To replace the footer line in all slides with a simple slide count uncomment this line

%\setbeamertemplate{navigation symbols}{} % To remove the navigation symbols from the bottom of all slides uncomment this line
}

\usepackage{graphicx} % Allows including images
\usepackage{booktabs} % Allows the use of \toprule, \midrule and \bottomrule in tables

%----------------------------------------------------------------------------------------
%	TITLE PAGE
%----------------------------------------------------------------------------------------

\title[Dependent Types]{Dependent Type Theory} % The short title appears at the bottom of every slide, the full title is only on the title page

\author{Cameron Wong} % Your name
\institute[98-317] % Your institution as it will appear on the bottom of every slide, may be shorthand to save space
{98-317 Hype For Types}
\date{2020-04-16} % Date, can be changed to a custom date

\begin{document}

\begin{frame}
\titlepage % Print the title page as the first slide
\end{frame}

%\begin{frame}
%\frametitle{Overview} % Table of contents slide, comment this block out to remove it
%\tableofcontents % Throughout your presentation, if you choose to use \section{} and \subsection{} commands, these will automatically be printed on this slide as an overview of your presentation
%\end{frame}

%----------------------------------------------------------------------------------------
%	PRESENTATION SLIDES
%----------------------------------------------------------------------------------------

%------------------------------------------------

\begin{frame}
  \frametitle{Last Time}

  \begin{itemize}
    \item Types \emph{depend} on values
    \begin{itemize}
      \item \texttt{('a, n) vec} as the type of length-$n$ lists
      \item \texttt{n fin} as the type of naturals less than $n$
    \end{itemize}
  \end{itemize}

\end{frame}

%------------------------------------------------

\begin{frame}
  \frametitle{Refinements}
  \begin{itemize}
    \item \emph{Lift} all values up to the type level
    \item Instead of complicated encodings like using \emph{succ} and
      \emph{fin} as type-level functions, just refer to values in types
    \item \texttt{nth: ('a,n) vec -> \{x:nat | x < n\} -> 'a}
  \end{itemize}
\end{frame}

%------------------------------------------------

\begin{frame}
  \frametitle{Refinements}
  Can also bind arguments, eg
  \begin{itemize}
    \item \texttt{repeat: \{n:int\} -> \{x:'a\} -> ('a,n) vec}
  \end{itemize}
\end{frame}

%------------------------------------------------

\begin{frame}
  \frametitle{Refinements}

  \begin{block}{Advantages}
    \begin{itemize}
      \item Very easy to understand
      \item Requires no fancy tricks like \texttt{'a fin}
    \end{itemize}
  \end{block}
\end{frame}

%------------------------------------------------

\begin{frame}
  \frametitle{Onto Theory}

  \textbf{Driving question}: \\
  What does it \emph{mean} for a type to depend on a value?
\end{frame}

%------------------------------------------------

\begin{frame}
  \frametitle{Onto Theory}

  \textbf{Driving question}: \\
  What \emph{is} the type \texttt{\{ x:t | p(x) \}}?
\end{frame}

%------------------------------------------------

\begin{frame}
  \frametitle{What is a refined type?}

  \phantom{Types = Sets?}
\end{frame}

%------------------------------------------------

\begin{frame}
  \frametitle{What is a refined type?}

  {Types = Sets?}
\end{frame}

%------------------------------------------------

\begin{frame}
  \frametitle{Types as sets}

  \begin{itemize}
    \item \texttt{nat} = $\mathbb{N}$
    \item \texttt{int} = $\mathbb{Z}$
    \item $\tau_1 \rightarrow \tau_2$ = (set-theoretic function)
    \item $\tau$\texttt{ list} = $\mathbb{N} \rightarrow \overline\tau$
  \end{itemize}
\end{frame}

%------------------------------------------------

\begin{frame}
  \frametitle{Types as sets}

  \begin{itemize}
    \item Refinements become very simple -- just use set comprehension!
    \begin{itemize}
      \item \texttt{\{x:t | p(x)\}} = $\{ x \in T \mid p(x) \}$
    \end{itemize}
  \end{itemize}
\end{frame}

%------------------------------------------------

\begin{frame}
  \frametitle{Types as sets}

  \begin{block}{Advantages}
    \begin{itemize}
      \item Very intuitive
      \item Can apply existing set theory research to type theory
    \end{itemize}
  \end{block}
\end{frame}

%------------------------------------------------

\begin{frame}
  \frametitle{Types as sets}

  \begin{block}{Disadvantages}
    \begin{itemize}
      \item Well...
    \end{itemize}
  \end{block}
\end{frame}

%------------------------------------------------

\begin{frame}[fragile]
  \frametitle{Types as sets}

  \begin{code}
    datatype t = T of t -> bool
  \end{code}
\end{frame}

%------------------------------------------------

\begin{frame}
  \frametitle{Types as sets}

  \begin{itemize}
    \item Let $S$ be the set representing the type \texttt{t}
    \item Certainly, $|S| = |S \rightarrow \texttt{bool}|$
  \end{itemize}
\end{frame}

%------------------------------------------------

\begin{frame}
  \frametitle{Types as sets}

  \begin{block}{Cantor's Theorem}
    For any set $A$, $|A| < |\mathcal{P}(A)|$.
  \end{block}
\end{frame}

%------------------------------------------------

\begin{frame}
  \frametitle{Types as sets}

  \begin{itemize}
    \item $S \rightarrow \texttt{bool}$ is equivalent to $\mathcal{P}(S)$
    \item Uh-oh...
  \end{itemize}
\end{frame}

%------------------------------------------------

\begin{frame}
  \frametitle{Types as sets}

  \begin{block}{Disadvantages}
    \begin{itemize}
      \item It's unsound!
    \end{itemize}
  \end{block}
\end{frame}

%------------------------------------------------

\begin{frame}
  \frametitle{Types as sets}
\end{frame}

%------------------------------------------------

\begin{frame}
  \frametitle{What is a refined type?}

  Recall: Curry-Howard Isomorphism
\end{frame}

%------------------------------------------------

\begin{frame}
  \frametitle{What is a refined type?}

  \begin{block}{Curry-Howard Isomorphism}
    Types are propositions, programs are proofs
  \end{block}
\end{frame}

%------------------------------------------------

\begin{frame}
  \frametitle{Types as propositions}

  \begin{block}{Review}
    Algebraic types (+ functions) correspond to \emph{propositional logic}
    (or \emph{zeroth-order logic}):
    \begin{itemize}
      \item $P \land Q$ corresponds to $A \times B$
      \item $P \lor Q$ corresponds to $A + B$
      \item $P \Rightarrow Q$ corresponds to $A \rightarrow B$
    \end{itemize}
  \end{block}
\end{frame}

%------------------------------------------------

\begin{frame}
  \frametitle{Types as propositions}

  What about \emph{first-order} logic?
\end{frame}

%------------------------------------------------

\begin{frame}
  \frametitle{Quantification}

  \begin{itemize}
    \item $\exists (x:\tau) . p(x)$
    \item $\forall (x:\tau) . p(x)$
  \end{itemize}
\end{frame}

%------------------------------------------------

\begin{frame}
  \frametitle{Quantification}

  For any $x:\tau$, $p(x)$ is a \emph{proposition}.
\end{frame}

%------------------------------------------------

\begin{frame}
  \frametitle{Quantification}

  For any $x:\tau$, $p(x)$ is a \sout{proposition} \emph{type}.
\end{frame}

%------------------------------------------------

\begin{frame}
  \frametitle{Quantification}

  $p$ is a function $\tau \rightarrow \texttt{type}$
\end{frame}

%------------------------------------------------

\begin{frame}
  \frametitle{Quantification}
\end{frame}

%------------------------------------------------

\begin{frame}
  \frametitle{Existentials}

  How to prove $\exists (x:\tau).p(x)$?
\end{frame}

%------------------------------------------------

\begin{frame}
  \frametitle{Existentials}

  Need:
  \begin{itemize}
    \item Some value $v:\tau$
    \item A proof of the proposition $p(v)$
  \end{itemize}
\end{frame}

%------------------------------------------------

\begin{frame}
  \frametitle{Existentials}

  Need:
  \begin{itemize}
    \item A value $v:\tau$
    \item A \sout{proof} program of \sout{the proposition} type $p(v)$
  \end{itemize}
\end{frame}

%------------------------------------------------

\begin{frame}
  \frametitle{Existentials}

  Need:
  \begin{itemize}
    \item A \sout{value} expression $v:\tau$
    \item A \sout{proof} \sout{program} expression of \sout{the proposition} type
      $p(v)$
  \end{itemize}
\end{frame}

%------------------------------------------------

\begin{frame}
  \frametitle{Existentials}

  A pair of expressions is a tuple!
\end{frame}

%------------------------------------------------

\begin{frame}
  \frametitle{Existentials}

  Dependent tuple: $\Sigma (x:\tau) . p(x)$

  \begin{prooftree}
    \AxiomC{$\Gamma \vdash e_1:\tau$}
    \AxiomC{$\Gamma \vdash e_2:p(e_1)$}
    \BinaryInfC{$\Gamma \vdash \langle e_1, e_2 \rangle
      : \Sigma (x:\tau) . p(x) $}
  \end{prooftree}

  \begin{prooftree}
    \AxiomC{$\Gamma \vdash e : \Sigma (x:\tau) . p(x) $}
    \UnaryInfC{$\Gamma \vdash \pi_1 e: \tau$}
  \end{prooftree}

  \begin{prooftree}
    \AxiomC{$\Gamma \vdash e : \Sigma (x:\tau) . p(x) $}
    \UnaryInfC{$\Gamma \vdash \pi_2 e: p(\pi_1 e)$}
  \end{prooftree}
\end{frame}

%------------------------------------------------

\begin{frame}
  \frametitle{Existentials}

  \textbf{Observation:} \\
  If $p(x) = \tau_2$ is a constant function, then $\Sigma (x:\tau_1) . p(x)$
  is the same as $\tau_1 \times \tau_2$
\end{frame}

%------------------------------------------------

\begin{frame}
  \frametitle{Existentials}

  \textbf{Observation:} \\
  $\tau_1 \times \tau_2$ is ``$\tau_2$ added $\tau_1$ times''
\end{frame}

%------------------------------------------------

\begin{frame}
  \frametitle{Quantification}

  \begin{itemize}
    \item $\Sigma(x:\tau).p(x)$ corresponds to $\exists (x:\tau) . p(x)$
    \item \phantom{$\Pi(x:\tau).p(x)$}
      corresponds to $\forall (x:\tau) . p(x)$
  \end{itemize}
\end{frame}

%------------------------------------------------

\begin{frame}
  \frametitle{Universals}

  What is a proof of $\forall (x:\tau) . p(x)$?
\end{frame}

%-----------------------------------------------

\begin{frame}
  \frametitle{Universals}

  Given a value $v:\tau$, produce a proof of the proposition $p(v)$
\end{frame}

%-----------------------------------------------

\begin{frame}
  \frametitle{Universals}

  Given a value $v:\tau$, produce a \sout{proof} expression of \sout{the proposition}
  type $p(v)$
\end{frame}

%-----------------------------------------------

\begin{frame}
  \frametitle{Universals}

  This is a \emph{function} of type $\tau \rightarrow p(v)$
\end{frame}

%-----------------------------------------------

\begin{frame}
  \frametitle{Universals}

  Dependent function: $\Pi(x:\tau). p(x)$

  \begin{prooftree}
    \AxiomC{$\Gamma, x:\tau \vdash e : p(x)$}
    \UnaryInfC{$\Gamma \vdash \lambda(x:\tau).e : \Pi(x:\tau).p(x)$}
  \end{prooftree}

  \begin{prooftree}
    \AxiomC{$\Gamma \vdash e_1 : \Pi(x:\tau).p(x)$}
    \AxiomC{$\Gamma \vdash e_2 : \tau$}
    \BinaryInfC{$\Gamma \vdash e_1\ e_2 : p(e_1)$}
  \end{prooftree}
\end{frame}

%-----------------------------------------------

\begin{frame}
  \frametitle{Universals}

  \textbf{Observation}: \\
  If $p(x) = \tau_2$, then $\Pi(x:\tau_1).p(x)$ is equivalent to $\tau_1
  \rightarrow \tau_2$
\end{frame}

%------------------------------------------------

\begin{frame}
  \frametitle{Quantification}

  \begin{itemize}
    \item $\Sigma(x:\tau).p(x)$ corresponds to $\exists (x:\tau) . p(x)$
    \item $\Pi(x:\tau).p(x)$ corresponds to $\forall (x:\tau) . p(x)$
  \end{itemize}
\end{frame}

%------------------------------------------------

\begin{frame}
  \frametitle{Refinements}

  \begin{block}{Back to refinements}
    What is \texttt{\{x:t | p(x)\}}?
  \end{block}
\end{frame}

%------------------------------------------------

\begin{frame}
  \frametitle{Refinements}

  \begin{itemize}
    \item \texttt{\{x:t | p(x)\}} is $\Sigma(x:t).p(x)$
    \item \texttt{\{x:t\} -> p(x)} is $\Pi(x:t).p(x)$
  \end{itemize}
\end{frame}

%------------------------------------------------

\begin{frame}
  \frametitle{Refinements}

  Note that regular functions can be subsumed by $\Pi$-types!

  \texttt{int -> int} $\rightsquigarrow$ $\Pi(\_:\texttt{int}).
  (\lambda \_:\texttt{int})$
\end{frame}

%------------------------------------------------

\begin{frame}
  \frametitle{What's in a proof?}

  \textbf{Next Question:} \\
  How to prove the proposition $p(x)$?
\end{frame}

%------------------------------------------------

\begin{frame}
  \frametitle{What's in a proof?}

  \textbf{Next Question:} \\
  How to \sout{prove the proposition} write a program of type $p(x)$?
\end{frame}

%------------------------------------------------

\begin{frame}
  \frametitle{What's in a proof?}

  \textbf{Next Question:} \\
  How to \sout{prove the proposition} write a program of type $3 < 5$?
\end{frame}

\newcommand{\tNat}{\texttt{nat}}
\newcommand{\tInt}{\texttt{int}}

%------------------------------------------------

\begin{frame}
  \frametitle{What's in a proof?}

  What is the definition of $<_{\tNat}$?
\end{frame}

%------------------------------------------------

\begin{frame}[fragile]
  \frametitle{What's in a proof?}

  \begin{code}
    fun 0 < s(_) = true
      | _ < 0 = false
      | s(n) < s(m) = n < m
  \end{code}
\end{frame}

%------------------------------------------------

\begin{frame}
  \frametitle{What's in a proof?}

  \begin{align*}
    3 < 5 \rightsquigarrow 2 < 4 \rightsquigarrow 1 < 3 \rightsquigarrow 0 < 2
    \rightsquigarrow \texttt{true}
  \end{align*}
\end{frame}

%------------------------------------------------

\begin{frame}
  \frametitle{What's in a proof?}

  \begin{align*}
    \underbrace{3 < 5}_{\texttt{bool}}
    \rightsquigarrow \underbrace{2 < 4}_{\texttt{bool}}
    \rightsquigarrow \underbrace{1 < 3}_{\texttt{bool}}
    \rightsquigarrow \underbrace{0 < 2}_{\texttt{bool}}
    \rightsquigarrow \underbrace{\texttt{true}}_{\texttt{bool}}
  \end{align*}
\end{frame}

%------------------------------------------------

\begin{frame}
  \frametitle{What's in a proof?}

  \begin{align*}
    \underbrace{3 < 5}_{\texttt{type}}
    \rightsquigarrow \underbrace{2 < 4}_{\texttt{type}}
    \rightsquigarrow \underbrace{1 < 3}_{\texttt{type}}
    \rightsquigarrow \underbrace{0 < 2}_{\texttt{type}}
    \rightsquigarrow \underbrace{\top}_{\texttt{type}}
  \end{align*}
\end{frame}

%------------------------------------------------

\begin{frame}
  \frametitle{What's in a proof?}

  \begin{block}{Curry-Howard}
    \begin{itemize}
      \item The type \texttt{unit} (or $\mathbf{1}$) corresponds to the
        proposition $\top$ (true)
      \item The type \texttt{void} (or $\mathbf{0}$) corresponds to the
        proposition $\bot$ (false)
    \end{itemize}
  \end{block}
\end{frame}

%------------------------------------------------

\begin{frame}
  \frametitle{What's in a proof?}

  The \emph{type} $3<5$ is equivalent to \texttt{unit}!
\end{frame}

%------------------------------------------------

\begin{frame}
  \frametitle{What's in a proof?}

  \begin{align*}
    \texttt{Refl}:(3<5)
  \end{align*}
\end{frame}

%------------------------------------------------

\begin{frame}
  \frametitle{What's in a proof?}

  \texttt{Refl} = ``true by definition''
\end{frame}

%------------------------------------------------

\begin{frame}
  \frametitle{What's in a proof?}

  \phantom{For usability:}
  \begin{align*}
    \texttt{(3, Refl) : \{x:int | x < 5\}}
  \end{align*}
\end{frame}

%-----------------------------------------------

\begin{frame}
  \frametitle{What's in a proof?}

  For usability:
  \begin{align*}
    \texttt{3 : \{x:int | x < 5\}}
  \end{align*}
\end{frame}

%------------------------------------------------

\begin{frame}[fragile]
  \frametitle{Example}

  \begin{code}
    type _ vec = [] : 'a vec
               | (::) : 'a * 'a vec -> 'a vec

    (* repeat : int -> 'a -> 'a vec *)
    fun repeat = _
  \end{code}
\end{frame}

%------------------------------------------------

\begin{frame}[fragile]
  \frametitle{Example}

  \begin{code}
    type _ vec = [] : ('a, 0) vec
               | (::) : 'a * ('a, n) vec -> ('a, n+1) vec

    (* repeat : {n:nat} -> {x:'a} -> {l:('a,n) vec} *)
    fun repeat = _
  \end{code}
\end{frame}

%------------------------------------------------

\begin{frame}[fragile]
  \frametitle{Example}

  \begin{code}
    type _ vec = [] : ('a, 0) vec
               | (::) : 'a * ('a, n) vec -> ('a, n+1) vec

    (* repeat : {n:nat} -> {x:'a} ->
       {l:('a,n) vec | forall (m < n) . nth m l = x}
     *)
    fun repeat = _
  \end{code}
\end{frame}

%------------------------------------------------

\begin{frame}[fragile]
  \frametitle{Example}

  \begin{align*}
    repeat : \Pi(n:\tNat) . \Pi(x:\alpha) . \Sigma(l:(\alpha,n)\ \texttt{vec})
    . \\
    \Pi(m: \Sigma (m':\tNat) . (m' < n)) . (\texttt{nth }l\ (\pi_1m) = x)
  \end{align*}
\end{frame}

%------------------------------------------------

\begin{frame}[fragile]
  \frametitle{Example}

  \begin{code}
    fun repeat 0 x = ([], fn (m,p) => _)
      | repeat n x = _
  \end{code}
\end{frame}

%------------------------------------------------

\newcommand{\tAbort}[1]{\texttt{abort}\ #1}

\begin{frame}[fragile]
  \frametitle{Example}

  $p$ has type $m < 0 \rightsquigarrow \bot$, so $p : \textbf{0}$
  \begin{code}
    fun repeat 0 x = ([], fn (m,p) => abort p)
      | repeat n x = _
  \end{code}
\end{frame}

%------------------------------------------------

\begin{frame}[fragile]
  \frametitle{Example}

  \begin{code}
    fun repeat 0 x = ([], fn (m,p) => abort p)
      | repeat n x =
          (* xs : ('a, n-1) vec
           * p : {m:nat | m < n} -> nth xs m = x
           *)
          let val (xs, p) = repeat (n-1) x
           in _
          end
  \end{code}
\end{frame}

%------------------------------------------------

\begin{frame}[fragile]
  \frametitle{Example}

  \begin{code}
    fun repeat 0 x = ([], fn (m,p) => abort p)
      | repeat n x =
          (* xs : ('a, n-1) vec
           * p : {m:nat | m < n} -> nth xs m = x
           *)
          let val (xs, p) = repeat (n-1) x
           (* _: {m:nat | m < n} -> (nth (x::xs) m = x) *)
           in (x::xs, _)
          end
  \end{code}
\end{frame}

%------------------------------------------------

\begin{frame}[fragile]
  \frametitle{Example}

  \begin{code}
    fun repeat 0 x = ([], fn (m,p) => abort p)
      | repeat n x =
          (* xs : ('a, n-1) vec
           * p : {m:nat | m < n-1} -> nth xs m = x
           *)
          let val (xs, p) = repeat (n-1) x
          (* p' : m < n *)
           in (x::xs, fn (0,p') => Refl (* defn of nth *)
                       | (m,p') => _)
          end
  \end{code}
\end{frame}

%------------------------------------------------

\begin{frame}[fragile]
  \frametitle{Example}

  \begin{code}
    fun repeat 0 x = ([], fn (m,p) => abort p)
      | repeat n x =
          (* xs : ('a, n-1) vec
           * p : {m:nat | m < n-1} -> nth xs m = x
           *)
          let val (xs, p) = repeat (n-1) x
          (* By definition of <, [p : m < n] is also a
           * proof of m-1 < n-1
           *)
           in (x::xs, fn (0,p') => Refl
                                   (* (m-1,p') is
                                    * Sigma(x:nat).(x<n-1)
                                    *)
                       | (m,p') => p(m-1, p'))
          end
  \end{code}
\end{frame}

%------------------------------------------------

\end{document}
