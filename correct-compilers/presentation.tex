%%%%%%%%%%%%%%%%%%%%%%%%%%%%%%%%%%%%%%%%%
% Beamer Presentation
% LaTeX Template
% Version 1.0 (10/11/12)
%
% This template has been downloaded from:
% http://www.LaTeXTemplates.com
%
% License:
% CC BY-NC-SA 3.0 (http://creativecommons.org/licenses/by-nc-sa/3.0/)
%
%%%%%%%%%%%%%%%%%%%%%%%%%%%%%%%%%%%%%%%%%

%----------------------------------------------------------------------------------------
%	PACKAGES AND THEMES
%----------------------------------------------------------------------------------------

\documentclass[usenames,dvipsnames]{beamer}

\usepackage{lstautogobble}
\usepackage{ulem}
\usepackage{bussproofs}
\usepackage{xcolor}
\definecolor{eclipseBlue}{RGB}{42,0,255}
\definecolor{eclipseGreen}{RGB}{63,127,95}

\mode<presentation> {

% The Beamer class comes with a number of default slide themes
% which change the colors and layouts of slides. Below this is a list
% of all the themes, uncomment each in turn to see what they look like.

%\usetheme{default}
%\usetheme{AnnArbor}
%\usetheme{Antibes}
%\usetheme{Bergen}
%\usetheme{Berkeley}
%\usetheme{Berlin}
%\usetheme{Boadilla}
\usetheme{CambridgeUS}
%\usetheme{Copenhagen}
%\usetheme{Darmstadt}
%\usetheme{Dresden}
%\usetheme{Frankfurt}
%\usetheme{Goettingen}
%\usetheme{Hannover}
%\usetheme{Ilmenau}
%\usetheme{JuanLesPins}
%\usetheme{Luebeck}
%\usetheme{Madrid}
%\usetheme{Malmoe}
%\usetheme{Marburg}
%\usetheme{Montpellier}
%\usetheme{PaloAlto}
%\usetheme{Pittsburgh}
%\usetheme{Rochester}
%\usetheme{Singapore}
%\usetheme{Szeged}
%\usetheme{Warsaw}

% As well as themes, the Beamer class has a number of color themes
% for any slide theme. Uncomment each of these in turn to see how it
% changes the colors of your current slide theme.

%\usecolortheme{albatross}
%\usecolortheme{beaver}
%\usecolortheme{beetle}
%\usecolortheme{crane}
%\usecolortheme{dolphin}
%\usecolortheme{dove}
%\usecolortheme{fly}
%\usecolortheme{lily}
\usecolortheme{orchid}
%\usecolortheme{rose}
%\usecolortheme{seagull}
%\usecolortheme{seahorse}
%\usecolortheme{whale}
%\usecolortheme{wolverine}

%\setbeamertemplate{footline} % To remove the footer line in all slides uncomment this line
%\setbeamertemplate{footline}[page number] % To replace the footer line in all slides with a simple slide count uncomment this line

%\setbeamertemplate{navigation symbols}{} % To remove the navigation symbols from the bottom of all slides uncomment this line
}

\usepackage{graphicx} % Allows including images
\usepackage{booktabs} % Allows the use of \toprule, \midrule and \bottomrule in tables
\usepackage{stmaryrd}

\newcommand{\Exp}{\texttt{Exp}}
\newcommand{\compile}{\texttt{compile}}
\newcommand{\syn}{\textsc{Syntax}}
\newcommand{\denot}[2][]{\llbracket #2 \rrbracket_{#1}}

\lstnewenvironment{haskell}[1][]
{\lstset{style=haskell,stringstyle=\color{orange},numbers=none,#1}}{}

%----------------------------------------------------------------------------------------
%	TITLE PAGE
%----------------------------------------------------------------------------------------

\title[]{Calculating Correct Compilers} % The short title appears at the bottom of every slide, the full title is only on the title page

\author{Cameron Wong} % Your name
\institute[CS-252R]{CS-252R}
\date{2022-11-03} % Date, can be changed to a custom date

\begin{document}

\begin{frame}
\titlepage % Print the title page as the first slide
\end{frame}

%\begin{frame}
%\frametitle{Overview} % Table of contents slide, comment this block out to remove it
%\tableofcontents % Throughout your presentation, if you choose to use \section{} and \subsection{} commands, these will automatically be printed on this slide as an overview of your presentation
%\end{frame}

%----------------------------------------------------------------------------------------
%	PRESENTATION SLIDES
%----------------------------------------------------------------------------------------

%------------------------------------------------

\begin{frame}
  \frametitle{}

  \begin{center}
    \Huge Calculating Correct Compilers
  \end{center}
\end{frame}

%------------------------------------------------

\begin{frame}
  \frametitle{}

  \begin{center}
    \Huge Calculating Correct \textcolor{red}{Compilers}
  \end{center}
\end{frame}

%------------------------------------------------

\begin{frame}
  \frametitle{Compilers?}

  \begin{itemize}
    \item Let $S$ and $T$ be the \emph{source} and \emph{target} languages
    \item We define a compiler to be a function
      \begin{equation}
        \compile : \syn_{S} \rightarrow \syn_{T}
      \end{equation}
  \end{itemize}
\end{frame}

%------------------------------------------------

\begin{frame}
  \frametitle{Compilers?}

  \begin{itemize}
    \item For our purposes, a compiler is a function from \emph{syntax} to
      \emph{syntax}

    \item Means we ignore practical concerns like linking, parsing, etc
  \end{itemize}
\end{frame}

%------------------------------------------------

\begin{frame}
  \frametitle{Compilers?}

  \begin{itemize}
    \item Also, assume that the input is well-formed
      \begin{itemize}
        \item No type errors, etc
      \end{itemize}
  \end{itemize}
\end{frame}

%------------------------------------------------

\begin{frame}
  \frametitle{Compilers?}

  \begin{itemize}
    \item As a last matter of terminology, the language that \compile{} is
      written in is called the \emph{host} language.
  \end{itemize}
\end{frame}

%------------------------------------------------

\begin{frame}
  \frametitle{}

  \begin{center}
    \Huge Calculating \textcolor{red}{Correct} Compilers
  \end{center}
\end{frame}

%------------------------------------------------

\begin{frame}
  \frametitle{Correct?}

  \begin{itemize}
    \item Languages are defined by their \emph{syntax} and \emph{semantics}
    \item If \compile{} acts on syntax, its \textit{correctness} should talk
      about semantics
  \end{itemize}
\end{frame}

%------------------------------------------------

\begin{frame}
  \frametitle{Correct?}

  \begin{itemize}
    \item How do we talk about a relationship between semantics?
  \end{itemize}
\end{frame}

%------------------------------------------------

\begin{frame}
  \frametitle{Correct?}

  \begin{itemize}
    \item Intuitively, $e$ and $\compile\ e$ should "mean the same thing"
    \item If we only worry about runtime, then $e$ and $\compile\ e$ should
      "\emph{do} the same thing"
  \end{itemize}
\end{frame}

%------------------------------------------------

\begin{frame}
  \frametitle{Correct?}

  \begin{itemize}
    \item This is actually very difficult to state formally!
  \end{itemize}
\end{frame}

%------------------------------------------------

\begin{frame}
  \frametitle{Correct?}

  \begin{itemize}
    \item What \emph{parts} of behavior need to be preserved?
      \begin{itemize}
        \item Performance?
        \item Memory?
      \end{itemize}
  \end{itemize}
\end{frame}

%------------------------------------------------

\begin{frame}
  \frametitle{Correct?}

  \begin{itemize}
    \item Even just looking at "return values" takes some machinery

    \item Suppose
      \begin{itemize}
        \item $e \rightsquigarrow^* v$ and
        \item $\compile\ e \rightsquigarrow^* \overline{v}$
      \end{itemize}

    \item $v$ is in $S$, but $\overline{v}$ is in $T$!

    \item For example, if $S$ is Java and $T$ is assembly, $v$ could be some
      a complex object!
  \end{itemize}
\end{frame}

%------------------------------------------------

\begin{frame}
  \frametitle{Correct?}

  \begin{itemize}
    \item One way is to define a relation $R$ on $\text{values}(S) \times
      \text{values}(T)$

    \item Can then define correctness as
      \begin{itemize}
        \item If $e \rightsquigarrow^* v$ and $\compile\ e \rightsquigarrow^*
          \overline{v}$, then $R(v, \overline{v})$
      \end{itemize}

    \item Could generalize and define the relation over $\syn_S \times \syn_T$,
      but this is typically much harder
  \end{itemize}
\end{frame}

%------------------------------------------------

\begin{frame}
  \frametitle{Correct?}

  \begin{itemize}
    \item Another way: define some common \emph{semantic domain} $D$, with
      \emph{denotation functions}
      \begin{itemize}
        \item $\denot[S]{\cdot} : \text{expressions}(S) \rightarrow D$
        \item $\denot[T]{\cdot} : \text{expressions}(T) \rightarrow D$
      \end{itemize}

    \item Then correctness is stated as
      \begin{equation}
        \denot[S]{e} = \denot[T]{\compile\ e}
      \end{equation}
  \end{itemize}

  \footnotetext{This is often called }
\end{frame}

%------------------------------------------------

\begin{frame}
  \frametitle{Correct?}

  \begin{itemize}
    \item For us: Our approach will look like the denotation method, embedded
      into the host language
      \begin{itemize}
        \item That is, $\denot[S]{\cdot}$ and $\denot[T]{\cdot}$ will be
          \emph{host language} functions, rather than purely metatheoretical
      \end{itemize}
  \end{itemize}
\end{frame}

%------------------------------------------------

\begin{frame}
  \frametitle{}

  \begin{center}
    \Huge \textcolor{red}{Calculating} Correct Compilers
  \end{center}
\end{frame}

%------------------------------------------------

\begin{frame}
  \frametitle{Setup}

  \begin{itemize}
    \item
  \end{itemize}
\end{frame}

\end{document}
