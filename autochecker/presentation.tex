%%%%%%%%%%%%%%%%%%%%%%%%%%%%%%%%%%%%%%%%%
% Beamer Presentation
% LaTeX Template
% Version 1.0 (10/11/12)
%
% This template has been downloaded from:
% http://www.LaTeXTemplates.com
%
% License:
% CC BY-NC-SA 3.0 (http://creativecommons.org/licenses/by-nc-sa/3.0/)
%
%%%%%%%%%%%%%%%%%%%%%%%%%%%%%%%%%%%%%%%%%

%----------------------------------------------------------------------------------------
%	PACKAGES AND THEMES
%----------------------------------------------------------------------------------------

\documentclass[usenames,dvipsnames]{beamer}

\usepackage{lstautogobble}
\usepackage{ulem}
\usepackage{xcolor}
\definecolor{eclipseBlue}{RGB}{42,0,255}
\definecolor{eclipseGreen}{RGB}{63,127,95}

\mode<presentation> {

% The Beamer class comes with a number of default slide themes
% which change the colors and layouts of slides. Below this is a list
% of all the themes, uncomment each in turn to see what they look like.

%\usetheme{default}
%\usetheme{AnnArbor}
%\usetheme{Antibes}
%\usetheme{Bergen}
%\usetheme{Berkeley}
%\usetheme{Berlin}
%\usetheme{Boadilla}
\usetheme{CambridgeUS}
%\usetheme{Copenhagen}
%\usetheme{Darmstadt}
%\usetheme{Dresden}
%\usetheme{Frankfurt}
%\usetheme{Goettingen}
%\usetheme{Hannover}
%\usetheme{Ilmenau}
%\usetheme{JuanLesPins}
%\usetheme{Luebeck}
%\usetheme{Madrid}
%\usetheme{Malmoe}
%\usetheme{Marburg}
%\usetheme{Montpellier}
%\usetheme{PaloAlto}
%\usetheme{Pittsburgh}
%\usetheme{Rochester}
%\usetheme{Singapore}
%\usetheme{Szeged}
%\usetheme{Warsaw}

% As well as themes, the Beamer class has a number of color themes
% for any slide theme. Uncomment each of these in turn to see how it
% changes the colors of your current slide theme.

%\usecolortheme{albatross}
%\usecolortheme{beaver}
%\usecolortheme{beetle}
\usecolortheme{crane}
%\usecolortheme{dolphin}
%\usecolortheme{dove}
%\usecolortheme{fly}
%\usecolortheme{lily}
%\usecolortheme{orchid}
%\usecolortheme{rose}
%\usecolortheme{seagull}
%\usecolortheme{seahorse}
%\usecolortheme{whale}
%\usecolortheme{wolverine}

%\setbeamertemplate{footline} % To remove the footer line in all slides uncomment this line
%\setbeamertemplate{footline}[page number] % To replace the footer line in all slides with a simple slide count uncomment this line

\setbeamertemplate{navigation symbols}{} % To remove the navigation symbols from the bottom of all slides uncomment this line
}

\usepackage{graphicx} % Allows including images
\usepackage{booktabs} % Allows the use of \toprule, \midrule and \bottomrule in tables
\usepackage{stmaryrd}
\usepackage{amsmath}
\usepackage{environ}
\usepackage{mathtools}
\usepackage{mathpartir}
\usepackage[absolute,overlay]{textpos}

\lstnewenvironment{code}[1][]
{\lstset{
  frame=none,
  xleftmargin=2pt,
  stepnumber=1,
  numbers=none,
  belowcaptionskip=\bigskipamount,
  captionpos=b,
  escapeinside={*'}{'*},
  language=Haskell,
  tabsize=2,
  emphstyle={\bf},
  commentstyle=\it,
  stringstyle=\mdseries\rmfamily,
  showspaces=false,
  keywordstyle=\bfseries\rmfamily,
  columns=flexible,
  basicstyle=\small\ttfamily,
  showstringspaces=false,
  morecomment=[l]\%,
  escapeinside={`}{`},
}}{}

\makeatletter
\def\blfootnote{\xdef\@thefnmark{}\@footnotetext}
\makeatother

\newcommand{\N}{\mathbb{N}}
\newcommand{\Exp}{\texttt{Exp}}
\newcommand{\Type}{\texttt{Ty}}
\newcommand{\Ctx}{\texttt{Ctx}}
\newcommand{\synth}{\texttt{synth}}
\renewcommand{\check}{\texttt{check}}
\newcommand{\Nat}{\texttt{Nat}}

%----------------------------------------------------------------------------------------
%	TITLE PAGE
%----------------------------------------------------------------------------------------

\title[]{Typechecker generation via equational reasoning}

\author{Cameron Wong} % Your name
\institute[CS-252R]{CS-252R final project}
\date{2022-12-01} % Date, can be changed to a custom date

\begin{document}

\begin{frame}
\titlepage % Print the title page as the first slide
\end{frame}

%\begin{frame}
%\frametitle{Overview} % Table of contents slide, comment this block out to remove it
%\tableofcontents % Throughout your presentation, if you choose to use \section{} and \subsection{} commands, these will automatically be printed on this slide as an overview of your presentation
%\end{frame}

%------------------------------------------------------------------------------
%	PRESENTATION SLIDES
%------------------------------------------------------------------------------

\begin{frame}
  \frametitle{Driving question}

  \begin{itemize}
    \item Can we systematically generate a typechecker from a collection of
      inference rules?
  \end{itemize}
\end{frame}

%------------------------------------------------

\begin{frame}
  \frametitle{Typechecking, the problem statement}

  \begin{itemize}
    \item Let \Exp{} be an arbitrary expression language with typing judgment
      \begin{equation*}
        \Gamma \vdash e : \tau
      \end{equation*}
    \item Two obvious ways to define a "typechecking" function:
      \begin{align}
        \check : \Ctx \rightarrow \Exp \rightarrow \Type \rightarrow \texttt{Bool} \\
        \synth : \Ctx \rightarrow \Exp \rightarrow \texttt{Option}\ \Type
      \end{align}
  \end{itemize}
\end{frame}

%------------------------------------------------

\begin{frame}
  \frametitle{Typechecking, the problem statement}

  \begin{itemize}
    \item What's our correctness theorem?

    \item "$\Gamma \vdash e : \tau$ if $\synth\ \Gamma\ e \equiv \texttt{Some}\ \tau$"
  \end{itemize}
\end{frame}

%------------------------------------------------

\begin{frame}[fragile]
  \frametitle{Theorem Prover Background}

  \begin{itemize}
    \item How to represent inference rules in the host language?

    \item In a proof assistant, typically done with an \emph{inductive family}:
      \begin{code}
        data `$\_\vdash\_:\_$` : `$\Ctx \rightarrow \Exp \rightarrow \Type \rightarrow\ $`Set
          typ-nat : `$\forall \{\Gamma$` n`$\} \rightarrow \Gamma \vdash $`Val `$n : \Nat$`
          typ-add : `$\forall \{\Gamma\ e_1\ e_2\} \rightarrow$`
            `$\Gamma \vdash e_1 : \Nat \rightarrow \Gamma \vdash e_1 : \Nat \rightarrow$` -- premises
            `$\Gamma \vdash e_1 + e_2 : \Nat$` -- conclusion
      \end{code}
  \end{itemize}
\end{frame}

%------------------------------------------------

\begin{frame}[fragile]
  \frametitle{Typechecking, the problem statement, in Agda}

  \begin{itemize}
    \item Correctness theorem, in Agda (some noise elided):
      \begin{code}
        synth-correct : `$\Gamma \vdash e : \tau \rightarrow \synth\ \Gamma\ e \equiv\ $`Some `$\tau$`
      \end{code}
  \end{itemize}
\end{frame}

%------------------------------------------------

\begin{frame}
  \frametitle{The Plan}

  \begin{itemize}
    \item Use equational reasoning to derive \texttt{synth-correct} and
      \texttt{synth} simultaneously

    \item Advantage is that we can perform induction on the judgment $\Gamma
      \vdash e : \tau$ instead of just structurally on $e$
  \end{itemize}
\end{frame}

%------------------------------------------------

\begin{frame}
  \frametitle{Source Language}

  \begin{align*}
    \tau &:= \Nat \mid \texttt{Bool} \mid \tau \rightarrow \tau \\
    e &:= true \mid false \mid \overline{n} \mid e + e \mid \lambda (x:\tau).e
  \end{align*}
  \begin{mathpar}
    \inferrule*[right=Typ-true]{\phantom{.}}{\Gamma \vdash true : \texttt{Bool}} \and
    \inferrule*[right=Typ-false]{\phantom{.}}{\Gamma \vdash false : \texttt{Bool}} \and
    \inferrule*[right=Typ-nat]{\phantom{.}}{\Gamma \vdash \overline{n} : \Nat} \and
    \inferrule*[right=Typ-add]
      {\Gamma \vdash e_1 : \Nat \and \Gamma \vdash e_2 : \Nat}
      {\Gamma \vdash e_1 + e_2 : \Nat} \and
    \inferrule*[right=Typ-abs]
      {\Gamma, x:\tau \vdash e : \tau'}
      {\Gamma \vdash \lambda (x:\tau).e : \tau \rightarrow \tau'}
  \end{mathpar}
\end{frame}

%------------------------------------------------

\begin{frame}
  \frametitle{The Derivation}

  \begin{itemize}
    \item Proceed by induction on the derivation of $\Gamma \vdash e : \tau$
  \end{itemize}
\end{frame}

%------------------------------------------------

{
  \addtobeamertemplate{frametitle}{}{%
    \textcolor{black}{Case $\textsc{Typ-true}$:}
  }
  \begin{frame}
    \frametitle{The Derivation}

    \begin{itemize}
      \item $e = true$, $\tau = \texttt{Bool}$
      \item Need proof of $\synth\ \Gamma\ true \equiv \texttt{Some Bool}$
      \item Currently, no clause of \synth{} for $true$!
        \begin{itemize}
          \item So define $\synth\ \Gamma\ true = \texttt{Some Bool}$
        \end{itemize}
    \end{itemize}
  \end{frame}
}

%------------------------------------------------

{
  \addtobeamertemplate{frametitle}{}{%
    \textcolor{black}{Case $\textsc{Typ-true}$:}
  }
  \begin{frame}[fragile]
    \frametitle{The Derivation}

    \begin{code}
      synth-correct `\textsc{Typ-true}` = begin
          `$\synth\ \Gamma\ true$`
        `$\equiv\langle \textnormal{Define clause: }\synth\ \Gamma\ true = \texttt{Some Bool} \rangle$`
          Some Bool
        `$\square$`
    \end{code}
  \end{frame}
}

%------------------------------------------------

{
  \addtobeamertemplate{frametitle}{}{%
    \textcolor{black}{Case $\textsc{Typ-add}$:}
  }
  \begin{frame}
    \frametitle{The Derivation}

    \begin{itemize}
      \item $e = e_1 + e_2$, $\tau = \Nat$
      \item Have $\Gamma \vdash e_1 : \Nat$, $\Gamma \vdash e_2 : \Nat$
      \item Need a proof of $\synth\ \Gamma\ (e_1+e_2) \equiv \Nat$
      \item Currently, no clause of \synth{} for $e_1+e_2$, so define
        $\synth\ \Gamma\ (e_1+e_2) = \Nat$
    \end{itemize}
  \end{frame}
}

%------------------------------------------------

{
  \addtobeamertemplate{frametitle}{}{%
    \textcolor{black}{Case $\textsc{Typ-add}$:}
  }
  \begin{frame}
    \frametitle{The Derivation}

    \begin{itemize}
      \item Wait, what?
    \end{itemize}
  \end{frame}
}

%------------------------------------------------

\begin{frame}
  \frametitle{Problem Statement, revisited}

  \begin{itemize}
    \item Need to forbid degenerate typecheckers!
    \item Corretness theorem is insufficiently precise -- need to talk about
      typechecker failure
  \end{itemize}
\end{frame}

%------------------------------------------------

\begin{frame}
  \frametitle{Problem Statement, revisited}

  \begin{itemize}
    \item Idea 1: New correctness theorem
      \begin{itemize}
        \item "$\Gamma \vdash e : \tau$ if \emph{and only if} $\synth
          \ \Gamma\ e \equiv \texttt{Some}\ \tau$"
      \end{itemize}
  \end{itemize}
\end{frame}

\end{document}
